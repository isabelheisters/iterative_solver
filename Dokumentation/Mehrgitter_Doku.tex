\documentclass[a4paper, 11pt]{article}
\usepackage[ngerman]{babel}
\begin{document}

    \tableofcontents

    \newpage

    \section{Einleitung}
        \subsection{Motivation}
        \subsection{Die Programmiersprache Julia}
    \section{Gegebenheiten}
    \section{Mathematisches Vorwissen}
            \subsection{Prolongation}
            \subsection{Restriktion}
            \subsection{Glaetter}
                \subsubsection{Jacobi}
                \subsubsection{Gauss Seidel}
            \subsection{Residuum}
            \subsection{Fouriermode}
    \section{Mehrgitter}
        \subsection{Was ist ein Mehrgitter?}
        \subsection{Warum benutzt man Mehrgitter?}
        \subsection{Zweigitter Algorithmus}
        \subsection{Mehrgitter Algorithmus}
    \section{Implementierung}
    \section{Parallelisierung}
    \section{Fazit}


    Hallo

\end{document}
